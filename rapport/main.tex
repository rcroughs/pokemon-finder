\documentclass[french]{article}

\usepackage[a4paper, total={6in, 8in}]{geometry}
\usepackage{babel}
\usepackage[utf8]{inputenc}
\usepackage[T1]{fontenc}

% Titres
% TODO: Ajouter les matricules
\title{Quel est ce Pokemon ? \\ Rapport}
\date{Année 2023-24}
\author{Antoine BERTHION \texttt{(000566199)} \and Romain CROUGHS \texttt{(000572261)} \and Lucas VAN PRAAG \texttt{(000569535)}}

\begin{document}
\maketitle

\section{Introduction}
Dans ce projet, nous devions faire de la recherche concurrente d'images à partir d'une image passée en paramètre, qui sera comparée avec
des images qui seront passées dans l'entrée standard. L'enjeu principal du projet est donc de créer deux processus, qui peuvent communiquer
avec le processus père afin d'effectuer la tâche précédemment énoncée.

Mais alors comment faire gérer et faire communiquer ces processus ?

Les systèmes d'exploitation modernes nous offrent beaucoup de méthodes pour coordonner nos processus. Dans ce rapport, nous décrirons comment
nous avons imaginé les processus et les protocoles de communication entre les processus. Enfin, nous parlerons des scripts bash qui sont utiles
au bon fonctionnement de notre programme.

\section{Programme \texttt{img-search}}
Ce programme est le coeur du projet. Il permet de récupérer les images à comparer, créer les processus fils qui vont effectuer la recherche concurrente
et enfin de récupérer les résultats des processus fils. 

\subsection{Récupération des paramètres}
\noindent Nous avons deux types de paramètres différents a récupérer:

\begin{itemize}
  \item Image que l'on souhaite comparer avec la banque d'images qui est passée en argument du programme. 
  \item Banque d'images qui est passée dans l'entrée standard.
\end{itemize}

Premièrement, récupérer l'image passée dans les paramètres est une tâche basique, car elle se trouve dans notre argument \texttt{argv[]} 
à l'index 1 (l'index 0 étant le nom de notre application).

Deuxièmement, pour récupérer l'entrée standard, nous pouvons utiliser l'appel systeme \texttt{int fgets(char buf, int buf\_size, stdin)} qui va lire
l'entrée standard ligne par ligne et mettre le contenu dans notre buffer et retourner \texttt{1}. Dans le cas où nous tombons sur un \texttt{EOF}, il renvoie 
simplement 0. Pour récupérer toutes les lignes de notre entrée standard, nous pouvons mettre cette fonction dans une boucle \texttt{while} et, par effet de bord,
la boucle va se bloquer quand le fichier sera terminé. 

\subsection{Création des processus enfant}
Pour gérer la recherche concurrente des images, nous devons créer deux processus qui vont avoir un fonctionnement similaire, mais va rechercher des images différentes
qui lui seront envoyées par le processus père via un pipe anonyme.

\subsection{Fonctionnement du père}
Notre processus père a une responsabilité: transmettre les paramètres entrés par l'utilisateur à ses deux enfants. Pour avoir une charge de travail équitable sur les 
deux processus enfants, nous tenons un compteur de combien d'images ont été entrées dans l'entrée standard. Si ce compteur est pair, il envoie l'image dans le pipe 
du premier enfant, dans le cas inverse, il l'envoie à l'autre enfant. 

Cependant, lors de l'écriture dans les pipes, nous avons fait face à la problématique suivante. La comparaison de distance entre deux images est un processus long. Au 
contraire, récupérer les images dans l'entrée standard ne l'est pas. Cela ne pose pas de problèmes quand nous utilisons une entrée standard qui est reliée au clavier 
de l'utilisateur, mais quand nous faisons un pipe entre nos processus (avec l'opérateur \texttt{|} en bash) le processus père écrit plus vite 
que le processus enfant ne peut le lire. Pour remédier à cela, nous avons opté pour une solution en signaux. Lorsque le père a écrit dans les deux pipes, au début 
de son exécution, il se met en pause avec un signal \texttt{SIGSTOP} qu'il s'envoie à lui même. Une fois qu'un des processus fils a fini son calcul il débloque 
l'exécution du père avec le signal \texttt{SIGCONT}. Cela permet aux pipes de ne jamais être surchargés. 

\subsection{Fonctionnement des processus enfants}
Le processus enfant va tout d'abord lire ce qu'il y a dans son pipe. Vu que le read est bloquant, nous n'avons qu'à attendre que le père écrive quelque chose pour 
commencer l'exécution. Une fois qu'une image est recue, nous allons créer un nouveau processus et modifier son fil d'exécution avec l'appel système \texttt{execlp} 
car nous avons veillé a ce que notre programme \texttt{img-dist} soit dans les variables \texttt{PATH} lors de l'execution de notre programme. Nous pouvons alors 
attendre qu'il se finisse avec la fonction \texttt{wait()} pour récupérer sa valeur de retour (qui est la distance entre l'image que l'on recherche et l'image 
de la banque d'image). Finalement, nous pouvons stocker ce résultat dans la mémoire partagée qui sera détaillé dans le chapitre suivant. 

\subsection{Stockage des resultats dans la mémoire partagée}
Pour stocker tous les résultats qui ont été calculés, nous pouvons les mettre dans un tableau dans la mémoire partagée. Cependant, nous devons veiller aux problèmes 
de concurrence, car nous ne savons pas quand les données vont être écrites par les deux processus fils. Pour éviter tous ces problèmes de concurrence, nous avons décidé
d'assigner deux places complètement différentes dans la mémoire, afin d'éviter tous les problèmes qui pourraient être causés par la concurrence des recherches. En pratique,
un enfant écrit et lit uniquement sa partie de la liste partagée et l'autre ne lit et écrit uniquement dans l'autre partie. Nous avons alors un tableau d'entier qui est 
de taille 2, et deux tableaux de caractère. Le tableau d'entier contiendra la distance la plus petite trouvée par chaque processus, et les tableaux de caractère le chemin 
correspondant à cette valeur. Une fois que toutes les recherches sont terminées, le processus père va comparer les deux éléments de la liste d'entiers, et le plus petite 
valeur sera le resultat de retour. La valeur par défaut de cette liste d'entier est -10, ce qui permet de savoir si aucune image n'a été entrée.  

\subsection{Gestion des signaux}
Enfin, pour bien coordonner nos processus, nous devions avoir une gestion habile des signaux. 

En ce qui concerne la gestion des signaux du fils, nous masquons les signaux qui peuvent bloquer son exécution afin de ne pas bloquer le calcul de la distance entre 
deux images. Nous allons alors uniquement traiter le signal \texttt{SIGUSR1} qui sera masqué pendant l'execution de \texttt{img-dist} et qui sera géré a la fin 
du calcul. Cela permet de gérer nous même les signaux et de ne pas avoir de arrêts avec des \texttt{SIGINT} qui pourraient survenir a des moments imprévus. 

En ce qui concerne le processus père, il va devoir gérer ses propres signaux et envoyer des signaux a ses enfants si besoin. Si il recoit une \texttt{SIGINT}, 
il va envoyer une signal \texttt{SIGUSR1} à ses enfants comme mentionné dans le paragraphe ci-dessus, et puis terminer sa propre execution. Il va avoir le même 
mecanisme s'il reçoit un signal \texttt{SIGPIPE} car, comme décrit dans les consignes, nous voulons que notre programme se termine si ce genre de signal 
se produit.

Enfin, nous utilisons les signaux pour savoir quand nous pouvons écrire dans les pipes. Une fois que le processus père a écrit dans les deux pipes, 
reliés a ses deux enfants, il se met en pause. Quand un enfant a libéré son pipe, il envoie un signal \texttt{SIGCONT} à son père afin que ce dernier se débloque 
et écrive l'adresse suivante dans le pipe. Cette pause se fait grace à la fonction standard \texttt{pause()} qui se met en pause un fil d'exécution et qui 
reprend cet exécution quand une signal est traité par un handler.

\section{Script bash \texttt{list-file}}
Le fonctionnement de ce programme est assez simple car il repose sur une fonction déjà existante en bash, le programme \texttt{find}. Nous allons lui ajouter les paramètres 
\texttt{-maxdepth 1} afin qu'on ne cherche pas les images dans les sous-dossiers qui peuvent se trouver dans le dossier de recherche. Également, nous allons inclure 
le paramètre \texttt{-type f} qui signifie que nous ne voulons que afficher les fichiers et non les dossiers. Enfin, le paramètre \texttt{-name ".*"} afin de ne pas 
inclure les fichiers masqués (ce paramètre n'est pas vraiment obligatoire mais il permet de rendre le programme plus robuste). Nous avons également veillé à 
informer l'utilisateur s'il manque un paramètre dans son appel au script. 

\section{Scipt bash \texttt{launcher}}
Finalement, nous devions créer un script qui permette de lancer notre programme. Nous avions donc deux modes à gérer, comme il est décrit dans l'énoncé. 

Premièrement, nous avons du modifier la variable \texttt{PATH}. Pour ce faire, nous devons comprendre comment fonctionne cette variable. Cette variable contient des 
adresses de fichiers contenant des programmes qui sont séparés par des :. Si nous voulons ajouter notre programme, il nous suffit d'ajouter notre chemin à cette 
variable, nous avons alors la commande \texttt{PATH="\$PATH:\$PWD/img-dist/"}. Nous incluons \texttt{PWD} car nous voulons le chemin absolu de notre pogramme. 

Pour gérer le mode interactif, c'est le plus simple, car nous appelons juste notre programme avec le fichier à rechercher en argument, et nous laissons l'utilisateur 
rentrer les chemins qu'il souhaite dans l'entrée standard. D'ailleurs, le calcul se fera dès qu'il passe une ligne, ce qui signifie que dès qu'il fera Ctrl+D, 
il aura le resultat. 

En ce concerne le mode automatique, nous allons utiliser notre script \texttt{list-file} et rediriger la sortie de ce script vers l'entrée 
de notre programme \texttt{img-search}, avec l'opérateur \texttt{|}. 

\section{Conclusion}
Pour conclure, nous avons pu, à travers ce projet, comprendre toutes les méthodes de communication entre processus. Entre les pipes utilisés pour communiquer du 
père vers les enfants, de la mémoire partagée pour enregistrer les résultats ou encore les signaux pour synchroniser les processus. Nous avons également pu 
aborder les possibles problèmes de concurrence dans la mémoire partagée.

\end{document}
