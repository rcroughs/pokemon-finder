\documentclass[french]{article}

\usepackage[a4paper, total={6in, 8in}]{geometry}
\usepackage{babel}
\usepackage[utf8]{inputenc}
\usepackage[T1]{fontenc}

% Titres
% TODO: Ajouter les matricules
\title{Quel est ce Pokemon ? \\ Rapport}
\date{Année 2023-24}
\author{Antoine BERTHION \texttt{(000566199)} \and Romain CROUGHS \texttt{(000572261)} \and Lucas VAN PRAAG \texttt{(000569535)}}

\begin{document}
\maketitle

\section{Introduction}
Dans ce projet, nous devions faire de la recherche concurrente d'images à partir d'une image passée en paramètre, qui sera comparée avec
des images qui seront passées dans l'entrée standard. L'enjeu principal du projet est donc de créer deux processus, qui peuvent communiquer
avec le processus père afin d'effectuer la tâche précédemment énoncée.

Mais alors comment faire gérer et faire communiquer ces processus ?

Les systèmes d'exploitation modernes nous offrent beaucoup de méthodes pour coordonner nos processus. Dans ce rapport, nous décrirons comment
nous avons imaginé les processus et les protocoles de communication entre les processus. Enfin, nous parlerons des scripts bash qui sont utiles
au bon fonctionnement de notre programme.

\section{Programme \texttt{img-search}}
Ce programme est le coeur du projet. Il permet de récupérer les images à comparer, créer les processus fils qui vont effectuer la recherche concurrente
et enfin de récupérer les résultats des processus fils. 

\subsection{Récupération des paramètres}
\noindent Nous avons deux types de paramètres différents a récupérer:

\begin{itemize}
  \item Image que l'on souhaite comparer avec la banque d'images qui est passé en argument du programme. 
  \item Banque d'image qui est passée dans l'entrée standard.
\end{itemize}

Premièrement, récupérer l'image passée dans les paramètres est une tâche basique, car elle se trouve dans notre argument \texttt{argv[]} 
à l'index 1 (l'index 0 étant le nom de notre application).

Deuxièmement, pour récupérer l'entrée standard, nous pouvons utiliser l'appel systeme \texttt{int fgets(char buf, int buf\_size, stdin)} qui va lire
l'entrée standard ligne par ligne et mettre le contenu dans notre buffer et retouner \texttt{1}. Dans le cas où nous tombons sur un \texttt{EOF}, il renvoie 
simplement 0. Pour récupérer toutes les lignes de notre entrée standard, nous pouvons mettre cette fonction dans une boucle \texttt{while} et, par effet de bord,
la boucle va se bloquer quand le fichier sera terminé. 

\subsection{Création des processus enfant}
Pour gérer la recherche concurrente des images, nous devons créer deux processus qui vont effectuer la même chose, mais sur des images différentes qui lui seront
envoyées par le processus père.
\end{document}
